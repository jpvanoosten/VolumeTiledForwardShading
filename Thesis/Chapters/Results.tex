% Chapter Template

\chapter{Results} % Main chapter title
\label{ch:Results}

\section{Introduction}

In this chapter, the performance of the various rendering technique described in Section \ref{sec:Algorithms} is analysed. The performance characteristics of traditional \emph{Forward Rendering} described in Section \ref{sec:ForwardRendering}, \emph{Tiled Forward Shading} described in Section \ref{sec:TiledForwardShading}, \emph{Volume Tiled Forward Shading} described in Section \ref{sec:VolumeTiledForwardShading}, and \emph{Volume Tiled Forward Shading with BVH Optimisation} described in Section \ref{sec:Optimization} are compared.

The performance results for the tests described in Chapter \ref{ch:ExperimentSetup} are shown and the performance results are analysed. All tests are performed at 1920x1080 resolution using NVidia GTX Titan X GPU. Timings shown are in milliseconds unless otherwise stated.

\section{Forward Rendering}

The \emph{Forward Rendering} technique was run with both the \emph{Sponza Atrium} \parencite{22_crytek_2017} scene and the \emph{San Miguel} \parencite{McGuire2011Data} scene mentioned in Section \ref{sec:Scenes}. Since Forward Rendering is not sensitive to the light density of the scene, the experiment was only executed with an increasing number of lights but the area in which the lights were spawned was not adjusted resulting in an increasing light density.

The following passes of the \emph{Forward Rendering} technique were captured:

\begin{enumerate}
\item{Depth Prepass}
\item{Opaque Pass}
\item{Transparent Pass}
\item{Total Rendering Time}
\end{enumerate}

\begin{figure}[H]
\centering
\includegraphics[width=\textwidth]{Charts/Forward_1920x1080_Sponza}
\decoRule
\caption{Sponza Atrium scene using \emph{Forward Rendering}.}
\label{fig:Forward_1920x1080_Sponza}
\end{figure}

\begin{table}[H]
\caption{Timings for rendering the Sponza Atrium scene using Forward Rendering.}
\label{tab:Forward_1920x1080_Sponza}
\centering
\begin{tabular}{*{5}{l}}
\toprule
\thead{Num \\Lights} & \thead{Depth \\Prepass} & \thead{Opaque \\Pass} & \thead{Transparent \\Pass} & \thead{Frame \\Time } \\
\midrule
0 & 0.148 & 0.237 & 0.065 & 0.450 \\
2 & 0.163 & 0.425 & 0.103 & 0.691 \\
4 & 0.168 & 0.656 & 0.151 & 0.975 \\
8 & 0.169 & 1.069 & 0.237 & 1.475 \\
16 & 0.170 & 1.903 & 0.385 & 2.459 \\
32 & 0.171 & 3.577 & 0.668 & 4.416 \\
64 & 0.171 & 6.905 & 1.281 & 8.356 \\
128 & 0.171 & 13.619 & 2.514 & 16.304 \\
256 & 0.171 & 27.286 & 4.991 & 32.449 \\
512 & 0.171 & 54.673 & 9.966 & 64.810 \\
1024 & 0.171 & 109.139 & 19.915 & 129.225 \\
\bottomrule\\
\end{tabular}
\end{table}

From these results, it can be observed that the \emph{Forward Rendering} technique exceeds the 60 FPS threshold after 128 lights and exceeds the 30 FPS threshold at just over 256 lights and the performance decreases linearly as the light count increases. In the case of the \emph{Sponza Atrium} scene the opaque rendering pass is the primary bottleneck for rendering while the transparent pass also degrades linearly but at a slower rate. This is expected behaviour for this scene since there are fewer transparent objects in the scene than opaque objects.

In the next experiment, the San Miguel scene is loaded and the number of lights in the scene is increased and the statistics captured.

\begin{figure}[H]
\centering
\includegraphics[width=\textwidth]{Charts/Forward_1920x1080_SanMiguel}
\decoRule
\caption{San Miguel scene using Forward Rendering.}
\label{fig:Forward_1920x1080_SanMiguel}
\end{figure}

\begin{table}[H]
\caption{Timings for rendering the San Miguel scene using Forward Rendering.}
\label{tab:Forward_1920x1080_SanMiguel}
\centering
\begin{tabular}{*{5}{l}}
\toprule
\thead{Num \\Lights} & \thead{Depth \\Prepass} & \thead{Opaque \\Pass} & \thead{Transparent \\Pass} & \thead{Frame \\Time } \\
\midrule
0 & 2.534 & 2.444 & 1.163 & 6.141 \\
2 & 2.630 & 2.640 & 1.463 & 6.733 \\
4 & 2.635 & 2.909 & 2.138 & 7.681 \\
8 & 2.631 & 3.666 & 3.522 & 9.819 \\
16 & 2.635 & 5.412 & 6.378 & 14.425 \\
32 & 2.625 & 9.156 & 12.224 & 24.005 \\
64 & 2.621 & 16.829 & 23.885 & 43.335 \\
128 & 2.529 & 32.557 & 51.023 & 86.109 \\
256 & 2.617 & 63.760 & 94.295 & 160.672 \\
\bottomrule\\
\end{tabular}
\end{table}

The San Miguel scene is clearly more complex than the Sponza scene and contains much more geometry. It also contains a considerable amount of transparent objects compared to the Sponza scene primarily due to the large trees in the middle of the hacienda as can be seen in Figure \ref{fig:SanMiguel_2}. Due to the large amount of transparent objects in this scene, the the transparent pass of the \emph{Forward Rendering} technique becomes the bottleneck of the rendering time. This is due to the large amount of overdraw that is caused by the transparent objects.

\begin{figure}[H]
\centering
\includegraphics[width=\textwidth]{Figures/SanMiguel_2}
\decoRule
\caption{San Miguel scene showing the two large trees in the middle of the hacienda. Each leaf on the tree is a transparent quad causing a lot of overdraw.}
\label{fig:SanMiguel_2}
\end{figure}

\section{Tiled Forward Shading}

In the next experiment, the \emph{Tiled Forward Shading} technique is tested. In the first test, the Sponza scene is loaded again and the lights are increased and the statistics collected. In the first case, the number of lights are increased while the size of the volume in which the lights are placed remain constant. In this case, the density of the lights increased linearly as the number of lights increased.

The following passes of the \emph{Tiled Forward Shading} technique were captured:

\begin{enumerate}
\item{Depth Prepass}
\item{Light Culling}
\item{Opaque Pass}
\item{Transparent Pass}
\item{Total Rendering Time}
\end{enumerate}

\begin{figure}[H]
\centering
\includegraphics[width=\textwidth]{Charts/TiledForward_1920x1080_Sponza}
\decoRule
\caption{Chart showing performance of rendering the Sponza scene using \emph{Tiled Forward Shading} with increasing light density.}
\label{fig:TiledForward_1920x1080_Sponza}
\end{figure}

\begin{table}[H]
\caption{Timings for rendering the Sponza scene using \emph{Tiled Forward Shading} with increasing light density.}
\label{tab:TiledForward_1920x1080_Sponza}
\centering
\begin{tabular}{*{6}{l}}
\toprule
\thead{Num \\Lights} & \thead{Depth \\Prepass} & \thead{Light \\Culling} & \thead{Opaque \\Pass} & \thead{Transparent \\Pass} & \thead{Frame \\Time } \\
\midrule
0 & 0.153 & 0.394 & 0.290 & 0.069 & 0.906 \\
2 & 0.160 & 0.525 & 0.289 & 0.069 & 1.043 \\
4 & 0.160 & 0.514 & 0.289 & 0.069 & 1.032 \\
8 & 0.160 & 0.530 & 0.290 & 0.069 & 1.049 \\
16 & 0.159 & 0.549 & 0.292 & 0.073 & 1.072 \\
32 & 0.157 & 0.568 & 0.296 & 0.080 & 1.100 \\
64 & 0.157 & 0.598 & 0.317 & 0.088 & 1.160 \\
128 & 0.157 & 0.618 & 0.367 & 0.115 & 1.257 \\
256 & 0.157 & 0.637 & 0.446 & 0.148 & 1.388 \\
512 & 0.163 & 0.711 & 0.565 & 0.195 & 1.633 \\
1024 & 0.163 & 0.768 & 0.781 & 0.239 & 1.951 \\
2048 & 0.168 & 1.202 & 1.596 & 0.597 & 3.563 \\
4096 & 0.170 & 1.922 & 2.681 & 1.041 & 5.814 \\
8192 & 0.170 & 3.348 & 5.010 & 1.904 & 10.433 \\
16384 & 0.169 & 6.026 & 9.725 & 3.747 & 19.666 \\
32768 & 0.169 & 11.679 & 19.123 & 7.055 & 38.026 \\
\bottomrule\\
\end{tabular}
\end{table}

The results reveal that the performance of the \emph{Tiled Forward Shading} technique improves on that of the \emph{Forward Rendering} technique shown in the previous section. Even with increasing light density, the \emph{Tiled Forward Shading} technique can handle approximately 14,300 dynamic lights before crossing the 60 FPS threshold and approximately 28,700 dynamic lights before cross the 30 FPS threshold.

As expected with increasing light density, the rendering time increases linearly as the number of lights increases. As with \emph{Forward Rendering}, the opaque pass increases at the highest rate compared to the transparent pass. The light culling technique also increases linearly as the number of lights increases but at a slower rate than the opaque rendering pass. 

In the next test, the same scene was used but the light density is not more than $1~light/unit^3$.

\begin{figure}[H]
\centering
\includegraphics[width=\textwidth]{Charts/TiledForward_1920x1080_Sponza_ConstantLightDensity}
\decoRule
\caption{Chart showing performance of rendering the Sponza scene using \emph{Tiled Forward Shading} with maximum light density of $1~light/unit^3$.}
\label{fig:TiledForward_1920x1080_Sponza_ConstantLightDensity}
\end{figure}


\begin{table}[H]
\caption{Timings for rendering the Sponza scene using \emph{Tiled Forward Shading} with maximum light density of $1~light/unit^3$.}
\label{tab:TiledForward_1920x1080_Sponza_ConstantLightDensity}
\centering
\begin{tabular}{*{6}{l}}
\toprule
\thead{Num \\Lights} & \thead{Depth \\Prepass} & \thead{Light \\Culling} & \thead{Opaque \\Pass} & \thead{Transparent \\Pass} & \thead{Frame \\Time } \\
\midrule
30,000 & 0.164 & 7.555 & 2.712 & 1.140 & 11.571 \\
50,000 & 0.166 & 12.609 & 2.768 & 1.139 & 16.683 \\
100,000 & 0.166 & 25.375 & 2.767 & 1.212 & 29.520 \\
200,000 & 0.166 & 50.633 & 2.684 & 1.155 & 54.637 \\
\bottomrule\\
\end{tabular}
\end{table}

In this case the time required for the opaque and transparent rendering passes remains constant. This shows that the \emph{Tiled Forward Shading} rendering technique is effective at improving the performance of the shading passes of the rendering technique when the light density remains constant. The major bottleneck of this rendering technique is the light culling pass, increasing linearly as the number of lights increases. This is an indication that in order to improve the performance of this rendering technique, the most obvious area to invest effort is the light culling pass.

\section{Volume Tiled Forward Shading}

The \emph{Volume Tiled Forward Shading} technique was tested with the Sponza scene. During the first test, the number of lights was increased without modifying the volume wherein lights were randomly placed. In the second test, the number of lights was increased while maintaining a maximum light density of $1~light/unit^3$. The rendering technique was also tested using the San Miguel scene while maintaining a maximum light density of $1~light/unit^3$.

Statistics were collected for the following passes of the rendering algorithm:

\begin{enumerate}
\item{Depth Prepass}
\item{Mark Active Tiles}
\item{Build Tile List}
\item{Assign Lights}
\item{Opaque Pass}
\item{Transparent Pass}
\item{Total Rendering Time}
\end{enumerate}

The \emph{Mark Active Tiles} pass will "activate" any volume tiles that contain a sample while the \emph{Build Tile List} pass creates a dense list of volume tile ID's that are activated. The \emph{Assign Lights} pass assigns lights to active tiles that are intersecting with the volume tiles. These passes are described in detail in Chapter \ref{ch:Implementation} but are mentioned here for clarity.

In the first test, the Sponza scene was loaded and the number of lights was increased while keeping the volume the lights were placed in a constant size resulting in an increasing light density.

\textbf{\begin{figure}[H]
\centering
\includegraphics[width=\textwidth]{Charts/VolumeTiledFoward_1920x1080_Sponza}
\decoRule
\caption{Chart showing performance of rendering the Sponza scene using \emph{Volume Tiled Forward Shading} with increasing light density.}
\label{fig:VolumeTiledFoward_1920x1080_Sponza}
\end{figure}
}

\begin{table}[H]
\caption{Timings for rendering the Sponza scene using \emph{Volume Tiled Forward Shading} with increasing light density. The \emph{Depth Prepass}, \emph{Mark Active Tiles}, and \emph{Build Tile List} data is omitted from this table to conserve space. }
\label{tab:VolumeTiledForward_1920x1080_Sponza}
\centering
\begin{tabular}{*{5}{l}}
\toprule
\thead{Num \\Lights} & \thead{Assign \\Lights} & \thead{Opaque \\Pass} & \thead{Transparent \\Pass} & \thead{Frame \\Time } \\
\midrule
0 & 0.136 & 0.296 & 0.078 & 0.798 \\
2 & 0.164 & 0.296 & 0.078 & 0.833 \\
4 & 0.164 & 0.296 & 0.078 & 0.833 \\
8 & 0.168 & 0.308 & 0.081 & 0.852 \\
16 & 0.169 & 0.303 & 0.080 & 0.849 \\
32 & 0.173 & 0.323 & 0.083 & 0.875 \\
64 & 0.172 & 0.309 & 0.081 & 0.858 \\
128 & 0.181 & 0.355 & 0.091 & 0.923 \\
256 & 0.197 & 0.394 & 0.100 & 0.992 \\
512 & 0.215 & 0.444 & 0.104 & 1.064 \\
1024 & 0.353 & 0.599 & 0.121 & 1.378 \\
2048 & 0.769 & 0.832 & 0.183 & 2.090 \\
4096 & 1.409 & 1.368 & 0.282 & 3.366 \\
8192 & 2.695 & 2.378 & 0.453 & 5.833 \\
16384 & 5.369 & 4.108 & 0.741 & 10.527 \\
32768 & 10.611 & 7.644 & 1.447 & 20.010 \\
65536 & 21.804 & 15.618 & 2.873 & 40.607 \\
\bottomrule\\
\end{tabular}
\end{table}

The \emph{Volume Tiled Forward Shading} rendering technique shows a significant performance increase over \emph{Tiled Forward Shading} even when the light density is increased. This is primarily due to the better light assignment that can be achieved by segmenting the light tiles in the depth. Since fewer false positives are generated by the volume tiled culling pass, the overall performance of the rendering technique is improved. It can be observed from these results that the time for the \emph{transparent} and \emph{opaque} passes does increase linearly, while the time required for the \emph{light assignment} stage increases at a faster rate.

In the next test, the \emph{Sponza} scene is rendered again using the \emph{Volume Tiled Forward Shading} technique. In this case, the light density does not exceed $1~light/unit^3$.

\textbf{\begin{figure}[H]
\centering
\includegraphics[width=\textwidth]{Charts/VolumeTiledFoward_1920x1080_Sponza_ConstantLightDensity}
\decoRule
\caption{Chart showing performance of rendering the Sponza scene using \emph{Volume Tiled Forward Shading} with maximum light density of $1~light/unit^3$.}
\label{fig:VolumeTiledFoward_1920x1080_Sponza_ConstantLightDensity}
\end{figure}
}

\begin{table}[H]
\caption{Timings for rendering the Sponza scene using \emph{Volume Tiled Forward Shading} with maximum light density of $1~light/unit^3$. The \emph{Depth Prepass}, \emph{Mark Active Tiles}, and \emph{Build Tile List} timings are omitted from this table to conserve space. }
\label{tab:VolumeTiledForward_1920x1080_Sponza}
\centering
\begin{tabular}{*{5}{l}}
\toprule
\thead{Num \\Lights} & \thead{Assign \\Lights} & \thead{Opaque \\Pass} & \thead{Transparent \\Pass} & \thead{Frame \\Time } \\
\midrule
30,000 & 9.135 & 1.419 & 0.279 & 11.114 \\
50,000 & 14.838 & 1.421 & 0.271 & 16.810 \\
100,000 & 30.901 & 1.440 & 0.266 & 32.894 \\
200,000 & 62.746 & 1.399 & 0.261 & 64.692 \\
500,000 & 200.852 & 1.356 & 0.252 & 202.739 \\
\bottomrule\\
\end{tabular}
\end{table}

While the average density of the lights in the scene does not exceed $1~light/unit^3$, the performance of the \emph{opaque} and \emph{transparent} shading passes remains almost insignificant to the overall performance of the rendering algorithm and it is clear from these results that the \emph{light assignment} pass consumes nearly $100\%$ of the total rendering time.

In the next test, the \emph{San Miguel} scene was rendered using the \emph{Volume Tiled Forward Shading} technique while maintaining a maximum light density of $1~light/unit^3$.

\textbf{\begin{figure}[H]
\centering
\includegraphics[width=\textwidth]{Charts/VolumeTiledFoward_1920x1080_SanMiguel_ConstantLightDensity}
\decoRule
\caption{Chart showing performance of rendering the San Miguel scene using \emph{Volume Tiled Forward Shading} with maximum light density of $1~light/unit^3$.}
\label{fig:VolumeTiledFoward_1920x1080_SanMiguel_ConstantLightDensity}
\end{figure}
}

\begin{table}[H]
\caption{Timings for rendering the San Miguel scene using \emph{Volume Tiled Forward Shading} with maximum light density of $1~light/unit^3$. The \emph{Depth Prepass}, \emph{Mark Active Tiles}, and \emph{Build Tile List} timings are omitted from this table to conserve space. }
\label{tab:VolumeTiledForward_1920x1080_Sponza}
\centering
\begin{tabular}{*{5}{l}}
\toprule
\thead{Num \\Lights} & \thead{Assign \\Lights} & \thead{Opaque \\Pass} & \thead{Transparent \\Pass} & \thead{Frame \\Time } \\
\midrule
0 & 0.257142 & 2.46628 & 1.1836 & 9.7806148 \\
2 & 0.325061 & 2.46049 & 1.17391 & 9.8930485 \\
4 & 0.330718 & 2.47328 & 1.19382 & 9.9534927 \\
8 & 0.325999 & 2.46624 & 1.18611 & 9.9100997 \\
16 & 0.320079 & 2.43724 & 1.17058 & 9.7923808 \\
32 & 0.310132 & 2.39869 & 1.14573 & 9.627357 \\
64 & 0.303072 & 2.34331 & 1.13876 & 9.4462456 \\
128 & 0.327978 & 2.25679 & 1.1598 & 9.4033059 \\
256 & 0.328362 & 2.33662 & 1.12413 & 9.4245832 \\
512 & 0.439239 & 2.54218 & 1.2427 & 10.3517244 \\
1024 & 0.764081 & 2.54836 & 1.23193 & 10.6700791 \\
2048 & 1.60282 & 2.50297 & 1.23063 & 11.3737761 \\
4096 & 2.96093 & 2.5123 & 1.22029 & 12.7359512 \\
8192 & 5.65014 & 2.50619 & 1.21972 & 15.3887541 \\
16384 & 10.5104 & 2.41291 & 1.17965 & 19.8892828 \\
32768 & 20.4104 & 2.39271 & 1.21395 & 29.7193525 \\
65536 & 48.5804 & 3.29217 & 2.76885 & 60.7786017 \\
131072 & 101.763 & 3.92572 & 3.90272 & 115.4876544 \\
\bottomrule\\
\end{tabular}
\end{table}

Due to the increased geometric complexity of the \emph{San Miguel} scene, the timings of the \emph{opaque} and \emph{transparent} shading passes are increased compared to that of the \emph{Sponza} scene but still remain relatively insignificant compared to the increasing timings of the \emph{light assignment} phase. Although the total rendering performance of \emph{Volume Tiled Forward Shading} technique is improved compared to that of the \emph{Tiled Forward Shading} technique, there is still plenty of room for improvement in the \emph{light assignment} phase.

In the next section a variation of the \emph{Volume Tiled Forward Shading} technique is tested. In this case, a Bounding Volume Hierarchy (BVH) is constructed over the lights in the scene before performing the light assignment pass.

\section{Volume Tiled Forward Shading (BVH)}

An optimized version of the \emph{Volume Tiled Forward Shading} technique was tested where a BVH was constructed over the lights before performing the \emph{light assignment} phase. Before the BVH can be constructed, the lights are sorted according to the Z-order of the lights as explained in Chapter \ref{ch:MortonCode} and \ref{ch:Sorting}. After constructing the BVH over the lights in the scene, the performance of the light assignment phase is significantly improved.

The passes of the Volume Tiled Forward Shading technique with BVH optimization are:

\begin{enumerate}
\item{Reduce Lights}
\item{Compute Morton Codes}
\item{Sort}
\item{Build BVH}
\item{Depth Prepass}
\item{Mark Active Tiles}
\item{Build Tile List}
\item{Assign Lights}
\item{Opaque Pass}
\item{Transparent Pass}
\item{Total Rendering Time}
\end{enumerate}

The first four stages are required to build the BVH over the lights in the scene and are unique to the \emph{Volume Tiled Forward Shading with BVH} technique. The last six stages of the \emph{Volume Tiled Forward Shading with BVH} technique are identical to that of the naïve \emph{Volume Tiled Forward Shading} technique. 

The first test using the \emph{Volume Tiled Forward Shading with BVH} technique renders the \emph{Sponza} scene with an increasing number of lights while maintaining a constant volume to position the lights. This results in an increasing light density.

\textbf{\begin{figure}[H]
\centering
\includegraphics[width=\textwidth]{Charts/VolumeTiledBVH_1920x1080_Sponza}
\decoRule
\caption{Chart showing performance of rendering the Sponza scene using \emph{Volume Tiled Forward Shading with BVH} with increasing light density.}
\label{fig:VolumeTiledBVH_1920x1080_Sponza}
\end{figure}
}


\begin{table}[H]
\caption{Timings for rendering the Sponza scene using \emph{Volume Tiled Forward Shading with BVH} with increasing light density. The \emph{Reduce Lights}, \emph{Compute Morton Codes}, \emph{Depth Prepass}, \emph{Mark Active Tiles}, and \emph{Build Tile List} timings are omitted from this table to conserve space. }
\label{tab:VolumeTiledBVH_1920x1080_Sponza}
\centering
\begin{tabular}{*{7}{l}}
\toprule
\thead{Num \\Lights} & \thead{Sort} & \thead{Build \\BVH} & \thead{Assign \\Lights} & \thead{Opaque \\Pass} & \thead{Transparent \\Pass} & \thead{Frame \\Time } \\
\midrule
1024 & 0.177 & 0.035 & 0.361 & 0.593 & 0.132 & 1.639 \\
2048 & 0.224 & 0.034 & 0.447 & 0.782 & 0.167 & 1.996 \\
4096 & 0.290 & 0.050 & 0.609 & 1.333 & 0.278 & 2.900 \\
8192 & 0.365 & 0.052 & 0.798 & 2.350 & 0.476 & 4.391 \\
16384 & 0.448 & 0.051 & 1.000 & 4.325 & 0.812 & 6.985 \\
32768 & 0.535 & 0.055 & 1.281 & 8.117 & 1.541 & 11.878 \\
65536 & 0.774 & 0.072 & 1.791 & 16.334 & 3.082 & 22.413 \\
131072 & 1.188 & 0.166 & 2.692 & 32.294 & 6.114 & 42.843 \\
\bottomrule\\
\end{tabular}
\end{table}

As expected with increasing light density, the \emph{opaque} shading pass consumes a significant portion of the rendering time. What is interesting is that the timings for the \emph{sorting} and the \emph{light assignment} phases remains almost constant during this test.

In the next test, the Sponza scene was rendered again this time maintaining a maximum light density of $1~light/unit^3$.

\textbf{\begin{figure}[H]
\centering
\includegraphics[width=\textwidth]{Charts/VolumeTiledBVH_1920x1080_Sponza_ConstantLightDensity}
\decoRule
\caption{Chart showing performance of rendering the Sponza scene using \emph{Volume Tiled Forward Shading with BVH} with maximum light density of $1~light/unit^3$.}
\label{fig:VolumeTiledBVH_1920x1080_Sponza_ConstantLightDensity}
\end{figure}
}

\begin{table}[H]
\caption{Timings for rendering the Sponza scene using \emph{Volume Tiled Forward Shading with BVH} with maximum light density of $1~light/unit^3$. The \emph{Reduce Lights}, \emph{Compute Morton Codes}, \emph{Depth Prepass}, \emph{Mark Active Tiles}, and \emph{Build Tile List} timings are omitted from this table to conserve space. }
\label{tab:VolumeTiledBVH_1920x1080_Sponza_ConstantLightDensity}
\centering
\begin{tabular}{*{7}{l}}
\toprule
\thead{Num \\Lights} & \thead{Sort} & \thead{Build \\BVH} & \thead{Assign \\Lights} & \thead{Opaque \\Pass} & \thead{Transparent \\Pass} & \thead{Frame \\Time } \\
\midrule
30,000 & 0.342 & 0.053 & 0.617 & 1.579 & 0.322 & 3.232 \\
50,000 & 0.474 & 0.100 & 0.599 & 1.515 & 0.289 & 3.299 \\
100,000 & 0.825 & 0.158 & 0.596 & 1.496 & 0.273 & 3.690 \\
200,000 & 1.361 & 0.230 & 0.643 & 1.448 & 0.273 & 4.327 \\
500,000 & 3.310 & 0.468 & 0.608 & 1.388 & 0.259 & 6.501 \\
1,000,000 & 5.691 & 0.840 & 0.681 & 1.385 & 0.264 & 9.501 \\
2,000,000 & 12.018 & 2.476 & 1.014 & 1.309 & 0.248 & 18.055 \\
3,000,000 & 16.199 & 4.160 & 0.911 & 1.354 & 0.270 & 24.246 \\
4,000,000 & 21.492 & 6.203 & 0.968 & 1.406 & 0.267 & 32.037 \\
5,000,000 & 29.818 & 8.297 & 0.937 & 1.376 & 0.273 & 42.754 \\
6,000,000 & 35.701 & 9.981 & 1.242 & 1.277 & 0.260 & 50.875 \\
7,000,000 & 41.238 & 11.817 & 1.069 & 1.311 & 0.255 & 58.452 \\
\bottomrule\\
\end{tabular}
\end{table}

In this test, the timings for the \emph{opaque} and \emph{transparent} shading passes remains constant and the primary overhead of the technique is consumed by the sorting and BVH construction phases.

In the next test, the \emph{San Miguel} scene is rendered using the \emph{Volume Tiled Forward Shading with BVH} technique while maintaining a maximum light density of $1~light/unit^3$.


\textbf{\begin{figure}[H]
\centering
\includegraphics[width=\textwidth]{Charts/VolumeTiledBVH_1920x1080_SanMiguel_ConstantLightDensity}
\decoRule
\caption{Chart showing performance of rendering the San Miguel scene using \emph{Volume Tiled Forward Shading with BVH} with maximum light density of $1~light/unit^3$.}
\label{fig:VolumeTiledBVH_1920x1080_SanMiguel_ConstantLightDensity}
\end{figure}
}

\begin{table}[H]
\caption{Timings for rendering the San Miguel scene using \emph{Volume Tiled Forward Shading with BVH} with maximum light density of $1~light/unit^3$. The \emph{Reduce Lights}, \emph{Compute Morton Codes}, \emph{Depth Prepass}, \emph{Mark Active Tiles}, and \emph{Build Tile List} timings are omitted from this table to conserve space. }
\label{tab:VolumeTiledBVH_1920x1080_Sponza_ConstantLightDensity}
\centering
\begin{tabular}{*{7}{l}}
\toprule
\thead{Num \\Lights} & \thead{Sort} & \thead{Build \\BVH} & \thead{Assign \\Lights} & \thead{Opaque \\Pass} & \thead{Transparent \\Pass} & \thead{Frame \\Time } \\
\midrule
500000 & 3.573 & 0.500 & 1.229 & 4.328 & 4.277 & 20.102 \\
1000000 & 6.157 & 0.899 & 1.607 & 6.301 & 7.106 & 28.410 \\
2000000 & 12.804 & 2.650 & 2.151 & 6.573 & 7.506 & 38.293 \\
3000000 & 17.382 & 4.456 & 3.981 & 10.297 & 12.952 & 56.069 \\
\bottomrule\\
\end{tabular}
\end{table}

In this case, the sorting continues to be the most expensive phase of the technique but the overhead of the opaque and transparent shading passes has a more noticeable effect with this scene. The fluctuations in the timings are primarily due to the difficulty in creating a good test scenario with this scene. The San Miguel scene is much larger than the Sponza scene. While trying to maintain a maximum light density, it was often the case that the lights were created outside of view of the camera, resulting in a light density of $0~light/unit^3$. The area in which the lights was created was adjusted and the timings were measured again but it was difficult to maintain consistent results due to fluctuations in the density of the lights. Since it is possible to determine a trend from these results, these fluctuations are deemed acceptable for this test.

What can be observed from these results is that although the performance of the \emph{Volume Tiled Forward Shading with BVH} technique still outperforms that of the naïve \emph{Volume Tiled Forward Shading} technique, the total frame time exceeds the acceptable limit of \SI{33.3}{\milli\second} at approximately 1.5 million light sources. This is primarily due to the increased number of volume tiles that are active in this scene due to the high number of transparent samples caused by the trees in the middle of the hacienda as can be seen in Figure \ref{fig:SanMiguel_2}.

\section{Techniques Compared}

In order to derive a general impression of the relative performance of each of the techniques discussed in this thesis, comparisons were made showing the performance of each technique in a single graph. The relative performance of rendering the Sponza scene with increasing light density is shown first.

\textbf{\begin{figure}[H]
\centering
\includegraphics[width=\textwidth]{Charts/Techniques_1920x1080_Sponza}
\decoRule
\caption{Chart showing relative performance of rendering the Sponza scene using \emph{Forward}, \emph{Tiled Forward}, \emph{Volume Tiled Forward}, and \emph{Volume Tiled Forward Shading with BVH} with an increasing light density.}
\label{fig:Techniques_1920x1080_Sponza}
\end{figure}
}

\begin{table}[H]
\caption{Timings showing relative performance of rendering the Sponza scene using \emph{Forward}, \emph{Tiled Forward}, \emph{Volume Tiled Forward}, and \emph{Volume Tiled Forward Shading with BVH} with an increasing light density. }
\label{tab:Techniques_1920x1080_Sponza}
\centering
\begin{tabular}{*{5}{l}}
\toprule
\thead{Num \\Lights} & \thead{Forward} & \thead{Tiled \\Forward} & \thead{Volume Tiled \\Forward} & \thead{Volume Tiled \\Forward (BVH)} \\
\midrule
0 & 0.450 & 0.906 & 0.798 & 0.808 \\
2 & 0.691 & 1.043 & 0.833 & 0.951 \\
4 & 0.975 & 1.032 & 0.833 & 0.954 \\
8 & 1.475 & 1.049 & 0.852 & 0.955 \\
16 & 2.459 & 1.072 & 0.849 & 0.956 \\
32 & 4.416 & 1.100 & 0.875 & 0.975 \\
64 & 8.356 & 1.160 & 0.858 & 0.976 \\
128 & 16.304 & 1.257 & 0.923 & 1.067 \\
256 & 32.449 & 1.388 & 0.992 & 1.095 \\
512 & 64.810 & 1.633 & 1.064 & 1.227 \\
1024 & 129.225 & 1.951 & 1.378 & 1.639 \\
2048 &  & 3.563 & 2.090 & 1.996 \\
4096 &  & 5.814 & 3.366 & 2.900 \\
8192 &  & 10.433 & 5.833 & 4.391 \\
16384 &  & 19.666 & 10.527 & 6.985 \\
32768 &  & 38.026 & 20.010 & 11.878 \\
65536 &  &  & 40.607 & 22.413 \\
131072 &  &  &  & 42.843 \\
\bottomrule\\
\end{tabular}
\end{table}

Even with increasing light density, both the naïve and optimized \emph{Volume Tiled Forward Shading} rendering techniques outperform \emph{Tiled Forward Shading}. This is primarily due to the improved light culling resulting in reduced false positives at geometric boundaries within a tile. It can be observed that even with a relatively low number of lights, the \emph{Volume Tiled Forward Shading} techniques outperforms \emph{Tiled Forward Shading}.

Next, the relative performance of the different rendering technique is compared when rendering the Sponza scene again but with a maximum light density of $1~light/unit^3$.

\begin{figure}[H]
\centering
\includegraphics[width=\textwidth]{Charts/Techniques_1920x1080_Sponza_ConstantLightDensity}
\decoRule
\caption{Chart showing relative performance of rendering the Sponza scene using \emph{Tiled Forward}, \emph{Volume Tiled Forward}, and \emph{Volume Tiled Forward Shading with BVH} with a maximum light density of $1~light/unit^3$. The performance of traditional \emph{Forward Rendering} is omitted from this chart because the timings start at 30,000 light sources which is already too many lights to gather any useful timings. }
\label{fig:Techniques_1920x1080_Sponza_ConstantLightDensity}
\end{figure}


\begin{table}[H]
\caption{Timings showing relative performance of rendering the Sponza scene using \emph{Tiled Forward}, \emph{Volume Tiled Forward}, and \emph{Volume Tiled Forward Shading with BVH} with a maximum light density of $1~light/unit^3$. }
\label{tab:Techniques_1920x1080_Sponza}
\centering
\begin{tabular}{*{4}{l}}
\toprule
\thead{Num \\Lights} & \thead{Tiled \\Forward} & \thead{Volume Tiled \\Forward} & \thead{Volume Tiled \\Forward (BVH)} \\
\midrule
30000 & 11.571 & 11.114 & 3.232 \\
50000 & 16.683 & 16.810 & 3.299 \\
100000 & 29.520 & 32.894 & 3.690 \\
200000 & 54.637 & 64.692 & 4.327 \\
500000 &  & 202.739 & 6.501 \\
1000000 &  &  & 9.501 \\
2000000 &  &  & 18.055 \\
3000000 &  &  & 24.246 \\
4000000 &  &  & 32.037 \\
5000000 &  &  & 42.754 \\
6000000 &  &  & 50.875 \\
7000000 &  &  & 58.452 \\
\bottomrule\\
\end{tabular}
\end{table}

While maintaining a maximum light density of $1~light/unit^3$ the rendering performance of the \emph{Volume Tiled Foward Shading with BVH} outperforms \emph{Tiled Forward} and naïve \emph{Volume Tiled Forward Shading} by far. What may be unexpected from these results is that the performance of the \emph{Tiled Forward Shading} technique outperforms that of the naïve \emph{Volume Tiled Forward Shading} technique. This is likely caused by the limited density of the lights in the scene resulting in fewer false positives at geometric boundaries mitigating the overhead of these false positives.

Next the performance characteristics of the various rendering techniques is analyzed while rendering the San Miguel scene.

\begin{figure}[H]
\centering
\includegraphics[width=\textwidth]{Charts/Techniques_1920x1080_SanMiguel_ConstantLightDensity}
\decoRule
\caption{Chart showing relative performance of rendering the San Miguel scene using \emph{Forward}, \emph{Tiled Forward}, \emph{Volume Tiled Forward}, and \emph{Volume Tiled Forward Shading with BVH} with a maximum light density of $1~light/unit^3$. }
\label{fig:Techniques_1920x1080_SanMiguel_ConstantLightDensity}
\end{figure}

\begin{table}[H]
\caption{Timings showing relative performance of rendering the San Miguel scene using \emph{Forward}, \emph{Tiled Forward}, \emph{Volume Tiled Forward}, and \emph{Volume Tiled Forward Shading with BVH} with a maximum light density of $1~light/unit^3$. }
\label{tab:Techniques_1920x1080_SanMiguel_ConstantLightDensity}
\centering
\begin{tabular}{*{5}{l}}
\toprule
\thead{Num \\Lights} & \thead{Forward} & \thead{Tiled \\Forward} & \thead{Volume Tiled \\Forward} & \thead{Volume Tiled \\Forward (BVH)} \\
\midrule
0 & 6.141 & 6.450 & 9.781 & 8.925 \\
2 & 6.733 & 7.032 & 9.893 & 9.942 \\
4 & 7.681 & 7.045 & 9.953 & 10.240 \\
8 & 9.819 & 7.030 & 9.910 & 9.756 \\
16 & 14.425 & 7.032 & 9.792 & 9.537 \\
32 & 24.005 & 7.036 & 9.627 & 9.595 \\
64 & 43.335 & 6.975 & 9.446 & 9.526 \\
128 & 86.109 & 6.713 & 9.403 & 9.420 \\
256 & 160.672 & 6.758 & 9.425 & 9.397 \\
512 &  & 6.759 & 10.352 & 10.606 \\
1024 &  & 7.139 & 10.670 & 10.539 \\
2048 &  & 7.604 & 11.374 & 10.663 \\
4096 &  & 8.495 & 12.736 & 10.837 \\
8192 &  & 10.041 & 15.389 & 10.781 \\
16384 &  & 20.531 & 19.889 & 10.893 \\
32768 &  & 46.051 & 29.719 & 11.644 \\
65536 &  & 88.118 & 60.779 & 12.631 \\
131072 &  & 139.965 & 115.488 & 13.154 \\
200000 &  &  &  & 14.050 \\
500000 &  &  &  & 20.102 \\
1000000 &  &  &  & 28.410 \\
2000000 &  &  &  & 38.293 \\
3000000 &  &  &  & 56.069 \\
\bottomrule\\
\end{tabular}
\end{table}

In this case the performance of the \emph{Volume Tiled Forward Shading with BVH} technique clearly outperforms any of the other techniques. An unexpected results of this test is that the performance of the \emph{Volume Tiled Forward Shading} technique exceeds that of \emph{Tiled Forward Shading} only after more than 16,384 active light sources in the scene. The improved performance of the \emph{Tiled Forward Shading} technique with less than 16,384 lights sources is due to the reduced cost of the \emph{light assignment} phase of the \emph{Tiled Forward Shading} technique that is a result of relatively fewer screen tiles than active volume tiles in the case of \emph{Volume Tiled Forward Shading}. At more than 16,384 lights, the benefits of the improved light culling of \emph{Volume Tiled Forward Shading} on the \emph{opaque} and \emph{transparent} rendering passes outweighs that of the \emph{Tiled Forward Shading} technique.

\section{Rate of Increase}

In order to provide an overall impression of the performance difference between the various techniques, the rate of increase ($r$) is computed using equation \ref{eqn:Rate_of_Increase}.

\begin{equation}
r=100\left(\frac{t_{max}-t_{min}}{l_{max}-l_{min}}\right)
\label{eqn:Rate_of_Increase}
\end{equation}

where $t_{max}$ is the maximum measured time, $t_{min}$ is the minimum measured time, $l_{max}$ is the maximum number of lights tested, and $l_{min}$ is the minimum number of lights tested. The rate of increase is multiplied by $100$ to shift the values into a meaningful range.

\begin{table}[H]
\caption{Relative rate of increase ($r$) for \emph{Forward Rendering} (\textbf{FR}), \emph{Tiled Forward Shading} (\textbf{TFS}), \emph{Volume Tiled Forward Shading} (\textbf{VTFS}), and \emph{Volume Tiled Forward Shading with BVH} (\textbf{VTFSBVH}). }
\label{tab:Rate_of_Increase}
\centering
\begin{tabular}{*{4}{l}}
\toprule
\thead{Technique} & \thead{Sponza} & \thead{Sponza \\($1~light/unit^3$)} & \thead{San Miguel \\($1~light/unit^3$)} \\
\midrule
\textbf{FR} & 12.576 & - & 60.363 \\
\textbf{TFS} & 0.113 & 0.025 & 0.102 \\
\textbf{VTFS} & 0.061 & 0.041 & 0.081 \\
\textbf{VTFSBVH} & 0.032 & 0.00079 & 0.00157 \\
\bottomrule\\
\end{tabular}
\end{table}


\section{Summary}

In summary, the performance of the \emph{Volume Tiled Forward Shading with BVH} technique outperforms all other techniques tested in this experiment when the number of lights in the scene exceeds 16,384, regardless of the light density and number of transparent objects in the scene. For scenes with less than 512 light sources, a simpler technique such as \emph{Tiled Forward Rendering} exhibits decent performance characteristics, but if there is a requirement to support a large number of active light sources in the scene, then it is recommended to invest the time and effort required to implement a rendering technique that optimizes the light assignment phase such as the \emph{Volume Tiled Forward Shading with BVH} technique described in this thesis.